\documentclass[10pt,draftclsnofoot,onecolumn]{IEEEtran}

\usepackage{setspace}
\usepackage{caption}

% *** GRAPHICS RELATED PACKAGES ***
\ifCLASSINFOpdf
  \usepackage[pdftex]{graphicx}
  % declare the path(s) where your graphic files are
  \graphicspath{images/}
  % and their extensions so you won't have to specify these with
  % every instance of \includegraphics
  \DeclareGraphicsExtensions{.pdf,.jpeg,.png}
\else
\fi

% correct bad hyphenation here
\hyphenation{op-tical net-works semi-conduc-tor}

\begin{document}
\pagenumbering{gobble}
\singlespacing
\title{Spatial Visualization\\ of Biodiversity}

\author{Ty~Skelton,
        Jasper~LaFortune,
        and~Alex~Shields}% <-this % stops a space

% The paper headers
\markboth{CS 461}%
{Spring 2016}

% make the title area
\maketitle

% As a general rule, do not put math, special symbols or citations
% in the abstract or keywords.
\begin{abstract} %
The Department of Integrative Biology at Oregon State has collected a large sample of biodiversity data from various sites in the Southwest United States.
Handling this data in its raw form requires certain technical knowledge of databases, as well as a bit of patience.
This presents a problem for biodiversity researchers and Department of Defense land managers, who need to be able to understand and make decisions about this data easily.
Our team addressed this problem by creating a web interface for spatially visualizing this data.
We enabled users to easily display useful graphs and maps about areas and species of interest, putting the information they care about most at their fingertips.

The Department of Defense made an investment in collecting all these samples so that it could better manage its land.
However, extracting meaning from that information is challenging for land managers and researchers alike.
Any good solution to this problem would allow users to easily access the information that is important to them.
Our solution provides an interface that allows users to select information of interest and see it in map and graph form.
Passerby at Expo will be invited to interact with our system and discover meaningful biodiversity patterns for themselves.
Oregon State biodiversity researchers have indicated that our product meets their needs.

\end{abstract}
\IEEEpeerreviewmaketitle

\newpage
\tableofcontents
\newpage
\pagenumbering{arabic}
\section{Introduction} % @TODO: Jasper
% Who requested it?
% Why was it requested?
% What is its importance?
% Who was/were your client(s)?
% Who are the members of your team?
% What were their roles?
% What was the role of the client(s)? (I.e., did they supervise only, or did they participate in doing development)

\section{Original Requirements} % @TODO: Ty
% This needs to be the original document, showing what you thought, at the time, was the project definition.
% This needs to include the original Gantt chart.

\section{Changes to Original Requirements} % @TODO: Alex
% What new requirements were added? What existing requirements were changed? What existing requirements were deleted? Why?
% Use the following table format:
% What was the final Gantt chart? @TODO: Ty

\section{Original Design Document} % @TODO: Alex
% Your original design document. Also, a discussion of what had to change over the course the year.

\section{Tech Review} % @TODO: Jasper
% Your original design document. Also, a discussion of what had to change over the course the year.

\section{Weekly Blog Posts} % @TODO: Ty, Alex, Jasper
% Your team weekly blog posts. These should be formatted nicely and clearly distinct from one another.

\section{Final Poster} % @TODO: Alex
% Your final poster, scaled and color-printed on a single 8.5"x11" paper. If you don't have access to a color printer, I will print it for you. Let me know.

\section{Project documentation} % @TODO: Ty
% How does your project work? @TODO: Jasper
%   What is its structure?
%   What is its Theory of Operation?
%   Block and flow diagrams are good here.
% How does one install your software, if any?
% How does one run it?
% Are there any special hardware, OS, or runtime requirements to run your software?
% Any user guides, API documentation, etc.

\section{New Technology} % @TODO: Alex
% What web sites were helpful? (Listed in order of helpfulness.)
% What, if any, reference books really helped?
% Were there any people on campus that were really helpful?

\section{What did we learn} % @TODO: Ty, Alex, Jasper
% What did you learn from all this? One per team member.
%   What technical information did you learn?
%   What non-technical information did you learn?
%   What have you learned about project work?
%   What have you learned about project management?
%   What have you learned about working in teams?
%   If you could do it all over, what would you do differently?
% Be honest here -- no B.S.
\end{document}
