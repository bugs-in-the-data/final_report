\documentclass[10pt,draftclsnofoot,onecolumn]{IEEEtran}

\usepackage{setspace}
\usepackage{caption}

% *** GRAPHICS RELATED PACKAGES ***
\ifCLASSINFOpdf
  \usepackage[pdftex]{graphicx}
  % declare the path(s) where your graphic files are
  \graphicspath{images/}
  % and their extensions so you won't have to specify these with
  % every instance of \includegraphics
  \DeclareGraphicsExtensions{.pdf,.jpeg,.png}
\else
\fi

% @TODO:
%   - go through and check that writing is in correct tense
%   - update screenshots
%   - add content to sections where relevant

% correct bad hyphenation here
\hyphenation{op-tical net-works semi-conduc-tor}

\begin{document}
\pagenumbering{gobble}
\singlespacing
\title{Spatial Visualization\\ of Biodiversity}

\author{Ty~Skelton,
        Jasper~LaFortune,
        and~Alec~Shields}% <-this % stops a space

% The paper headers
\markboth{CS 461}%
{Spring 2016}

% make the title area
\maketitle

% As a general rule, do not put math, special symbols or citations
% in the abstract or keywords.
\begin{abstract} % @NOTE: JASPER
The Department of Integrative Biology at Oregon State has collected a large sample of biodiversity data from various sites in the Southwest United States.
Handling this data in its raw form requires certain technical knowledge of databases, as well as a bit of patience.
This presents a problem for biodiversity researchers and Department of Defense land managers, who need to be able to understand and make decisions about this data easily.
Our team addressed this problem by creating a web interface for spatially visualizing this data.
We enabled users to easily display useful graphs and maps about areas and species of interest, putting the information they care about most at their fingertips.

The Department of Defense made an investment in collecting all these samples so that it could better manage its land.
However, extracting meaning from that information is challenging for land managers and researchers alike.
Any good solution to this problem would allow users to easily access the information that is important to them.
Our solution provides an interface that allows users to select information of interest and see it in map and graph form.
Passerby at Expo will be invited to interact with our system and discover meaningful biodiversity patterns for themselves.
Oregon State biodiversity researchers have indicated that our product meets their needs.

\end{abstract}
\IEEEpeerreviewmaketitle

\newpage
\pagenumbering{arabic}
% initial
\end{document}
