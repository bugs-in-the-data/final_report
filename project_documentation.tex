\subsection{How it works}
Our system consists of three main views.
The Filter View allows the user to select information of interest.
The Map View displays selected information on a map.
The Graph View shows graphs of the selected information.
When the user selects information of interest in the Filter View, the website posts these parameters to the server, as seen in Figure~\ref{fig:dataflow}.
A Django script on the server uses these parameters to form a database query.
The MySQL database returns the requested data to the server, which formats the data into the Map and Graph Views.

\begin{figure}[h]
\centering
\includegraphics[width=0.75\textwidth]{DataflowDiagram.png}
\captionsetup{justification=centering}
\caption{
  Dataflow diagram.
  The server forms filter parameters into a database request, the results of which are passed to the frontend views.
}
\label{fig:dataflow}
\end{figure}

\subsection{How to install it}
Installing our web app is actually pretty simple:
\begin{list}{-}{}
\item Install the \texttt{mod\_wsgi} package for Apache, which is a module that implements a compliant interface for hosting Python based web apps.
\item Install the Python package manager, Pip.
\item Using Pip, install the framework Django: \texttt{pip install Django}
\item Clone the code repository and install MySQL.
\item Finally, at the command-line run \texttt{./manage.py make migrations; ./manage.py migrate} to generate database tables based on our ORM.
\end{list}

\subsection{How to run it}
Running the server at the command line is as easy as typing \texttt{./manage.py runserver}.
To deploy it via apache, simply configure apache to point to the project folder, the \texttt{mod\_wsgi} installation, and the Python binary.
This will host it statically and will run the necessary scripts on every load of the page.

\subsection{System requirements}
As hinted at in the previous sections, there are certain technologies which are required to run this web app:
\begin{list}{-}{}
\item MySQL server
\item Django v1.6.11
\item Python 2 or higher
\item Apache
\item mod\_wsgi
\end{list}
