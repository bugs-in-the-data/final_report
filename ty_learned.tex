% What did you learn from all this? One per team member.
%   What technical information did you learn?
%   What non-technical information did you learn?
%   What have you learned about project work?
%   What have you learned about project management?
%   What have you learned about working in teams?
%   If you could do it all over, what would you do differently?
% Be honest here -- no B.S.
\par Contrary to what I expected going in, I ended up learning a lot through my capstone project.
Initially I assumed it'd be just another web development project on large set of data, which is something that I've done before many times.
Looking back I can say that I'm pleasantly surprised by how much I gained in this process and the experiences I had.

\par Honestly, this project taught me the importance of having a solid team.
A solid team isn't just people you enjoy to work with, which I did, but it means people that you can count on.
It's safe to say every student, computer science or otherwise, has a go-to horror story for some kind of group work project that they couldn't wait to be done with.
This year was a testament to the fact that good groups do exist and reinstilled my faith in group projects.
No matter the work, deadline, or curveball- I knew I could count on my team to get things done.

\par Outside of that I learned some cool new technology this term.
My primary gain in this area was definitely in the use of Python as a language for a standalone web app.
I've done plenty of web development in my time, heck it's been my primary job throughout college, but I still love to learn new things and new technologies.
The use of Python was a compromise between all members of the group, since some of us were familiar with web development and some of us were more familiar with Python as a language.
Since working on this project I've been able to talk about the experience and my work in Python as a web language and it's been very beneficial for me.

\par Lastly, the most stand-out thing that I learned this year I'd have to say is the technical writing skills and a proficiency in laTex.
We've done a ton of writing in this class and almost all of it was done through the use of laTex.
I've gotten much better at putting words to paper, so to speak, while being appropriately technical in my explanations of concepts.
laTex, however, has grown on me significantly.
I first learned it this year due to the requirement with this class and I have come to love it very much and use it for almost all writing projects.

\par Looking back on this project, the only thing I can say I wish we did differently is put more emphasis on our stretch goals.
This is hard to say, because it was a very tough year for all of us and we put the most we could into the project.
With hindsight being 20/20 and all it's easy to say we should have spent less time polishing lesser aspects, but in the moment it felt right and you can't fault us for that.
The stretch goals were to simply make cooler graphs and statistics, but the app came first and we wanted it to be our best. 
